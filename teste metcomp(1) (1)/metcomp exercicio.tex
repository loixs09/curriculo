%% start of file `template.tex'.
%% Copyright 2006-2015 Xavier Danaux (xdanaux@gmail.com), 2020-2022 moderncv maintainers (github.com/moderncv).
%
% This work may be distributed and/or modified under the
% conditions of the LaTeX Project Public License version 1.3c,
% available at http://www.latex-project.org/lppl/.


\documentclass[11pt,a4paper,sans]{moderncv}        % possible options include font size ('10pt', '11pt' and '12pt'), paper size ('a4paper', 'letterpaper', 'a5paper', 'legalpaper', 'executivepaper' and 'landscape') and font family ('sans' and 'roman')

% moderncv themes
\moderncvstyle{classic}                             % style options are 'casual' (default), 'classic', 'banking', 'oldstyle' and 'fancy'
\moderncvcolor{orange}                               % color options 'black', 'blue' (default), 'burgundy', 'green', 'grey', 'orange', 'purple' and 'red'
\usepackage[brazil]{babel}
%\renewcommand{\familydefault}{\sfdefault}         % to set the default font; use '\sfdefault' for the default sans serif font, '\rmdefault' for the default roman one, or any tex font name
%\nopagenumbers{}                                  % uncomment to suppress automatic page numbering for CVs longer than one page

% adjust the page margins
\usepackage[scale=0.9]{geometry}
\setlength{\footskip}{136.00005pt}                 % depending on the amount of information in the footer, you need to change this value. comment this line out and set it to the size given in the warning
%\setlength{\hintscolumnwidth}{3cm}                % if you want to change the width of the column with the dates
%\setlength{\makecvheadnamewidth}{10cm}            % for the 'classic' style, if you want to force the width allocated to your name and avoid line breaks. be careful though, the length is normally calculated to avoid any overlap with your personal info; use this at your own typographical risks...

% font loading
% for luatex and xetex, do not use inputenc and fontenc
% see https://tex.stackexchange.com/a/496643
\ifxetexorluatex
  \usepackage{fontspec}
  \usepackage{unicode-math}
  \defaultfontfeatures{Ligatures=TeX}
  \setmainfont{Latin Modern Roman}
  \setsansfont{Latin Modern Sans}
  \setmonofont{Latin Modern Mono}
  \setmathfont{Latin Modern Math} 
\else
  \usepackage[T1]{fontenc}
  \usepackage{lmodern}
\fi

% document language
\usepackage[brazil]{babel}  % FIXME: using spanish breaks moderncv

% personal data
\name{Lois}{Valentim}
%\title{Résumé title}                               % optional, remove / comment the line if not wanted
\born{9 outubro 2003}                                 % optional, remove / comment the line if not wanted
\address{Rua cambúba 194}{}{Rio de Janeiro}% optional, remove / comment the line if not wanted; the "postcode city" and "country" arguments can be omitted or provided empty
\phone[mobile]{+55~(85)~988414310}                   % optional, remove / comment the line if not wanted; the optional "type" of the phone can be "mobile" (default), "fixed" or "fax"
%\phone[fixed]{+2~(345)~678~901}
%\phone[fax]{+3~(456)~789~012}
\email{lois22@ov.ufrj.br}                               % optional, remove / comment the line if not wanted
%\homepage{www.johndoe.com}                         % optional, remove / comment the line if not wanted

% Social icons
%\social[linkedin]{john.doe}                        % optional, remove / comment the line if not wanted
%\social[xing]{john\_doe}                           % optional, remove / commFent the line if not wanted
%\social[twitter]{ji\_doe}                             % optional, remove / comment the line if not wanted
%\social[github]{jdoe}                              % optional, remove / comment the line if not wanted
%\social[gitlab]{jdoe}                              % optional, remove / comment the line if not wanted
% \social[stackoverflow]{0000000/johndoe}            % optional, remove / comment the line if not wanted
% \social[bitbucket]{jdoe}                           % optional, remove / comment the line if not wanted
% \social[skype]{jdoe}                               % optional, remove / comment the line if not wanted
% \social[orcid]{0000-0000-000-000}                  % optional, remove / comment the line if not wanted
% \social[researchgate]{jdoe}                        % optional, remove / comment the line if not wanted
% \social[researcherid]{jdoe}                        % optional, remove / comment the line if not wanted
% \social[telegram]{jdoe}                            % optional, remove / comment the line if not wanted
% \social[whatsapp]{12345678901}                     % optional, remove / comment the line if not wanted
% \social[signal]{12345678901}                       % optional, remove / comment the line if not wanted
% \social[matrix]{@johndoe:matrix.org}               % optional, remove / comment the line if not wanted
% \social[googlescholar]{googlescholarid}            % optional, remove / comment the line if not wanted


%\extrainfo{additional information}                 % optional, remove / comment the line if not wanted
%\photo[64pt][0.4pt]{placeholder.png}                       % optional, remove / comment the line if not wanted; '64pt' is the height the picture must be resized to, 0.4pt is the thickness of the frame around it (put it to 0pt for no frame) and 'picture' is the name of the picture file
%\quote{Some quote}                                 % optional, remove / comment the line if not wanted

% bibliography adjustments (only useful if you make citations in your resume, or print a list of publications using BibTeX)
%   to show numerical labels in the bibliography (default is to show no labels)
%\makeatletter\renewcommand*{\bibliographyitemlabel}{\@biblabel{\arabic{enumiv}}}\makeatother
\renewcommand*{\bibliographyitemlabel}{[\arabic{enumiv}]}
%   to redefine the bibliography heading string ("Publications")
%\renewcommand{\refname}{Articles}

% bibliography with mutiple entries
%\usepackage{multibib}
%\newcites{book,misc}{{Books},{Others}}
%----------------------------------------------------------------------------------
%            content
%----------------------------------------------------------------------------------

\begin{document}
%\begin{CJK*}{UTF8}{gbsn}                          % to typeset your resume in Chinese using CJK
%-----       resume       ---------------------------------------------------------
\makecvtitle

\section{Formação Acadêmica}
\cventry{2013--2017}{Ensino Fundamental completo}{Colégio Farias Brito }{Fortaleza}{}{}  
% arguments 3 to 6 can be left empty
\cventry{2019--2021}{Ensimo Médio completo}{Colégio Antares Fátima}{Fortaleza}{\textit{}}{}
\cventry{2022-- 2025}{Graduação}{Universidade Federal Do Rio de Janeiro}{Rio de Janeiro}{}{}
\cventry{2026-- Dias atuais}{Mestrado}{California Institute of Technology (Caltech) }{California}{Pesquisa direcionada para escala subatômica e cosmológica do processo de formação estelar}{}

%\section{Master thesis}
%\cvitem{title}{\emph{Title}}
%\cvitem{supervisors}{Supervisors}
%\cvitem{description}{Short thesis abstract}

\section{Produção}
\subsection{Projeto de Mestrado}
\cventry{3 anos}{Astrofísica de partículas}{Orientador : E. S. (Sterl) Phinney}{}{}{
\begin{itemize}
\item estudo sobre a física intrínseca em dados observacionais de simulações cosmológica hidrodinâmica em partículas de matéria escura 
\end{itemize}}

\subsection{Iniciação cientifíca }
    \cventry{3 anos}{Área de extragaláctica}{Orientador : Thiago Signorini}{}{}{
\begin{itemize}
\item Determinando a escala de tempo para cessação da formação estelar a partir de simulações e modelos teóricos
\end{itemize}}

\section{interesses pessoais}
\cvitem{Poetisa amadora }{Poemas publicados por concursos, os quais participei}
\cvitem{Basquete}{}
\cvitem{Bailarina}{O ballet presente em minha vida por 13 anos,com prêmios nacionais e internacionais}
\cvitem{curso}{piloto}

%\item Achievement 2 (with sub-achievements)

%  \begin{itemize}
 % \item 
 % \item Sub-achievement (b), with sub-sub-achievements (don't do this!);
  %  \begin{itemize}
   % \item Sub-sub-achievement i;
   % \item Sub-sub-achievement ii;
   % \item Sub-sub-achievement iii;
   % \end{itemize}
 % \item Sub-achievement (c);
 % \end{itemize}
%\item Achievement 3
%\item Achievement 4
%\end{itemize}}
%\cventry{year--year}{Job title}{Employer}{City}{}{Description line 1\newline{}Description line %\newline{}Description line 3}
%\subsection{Miscellaneous}
%\cventry{year--year}{Job title}{Employer}{City}{}{Description}

\section{Languages}
\cvitemwithcomment{}{}{}
\cvitemwithcomment{Português}{Bom}{}
\cvitemwithcomment{Inglês}{Bom}{}
\cvitemwithcomment{Espanhol}{bom}{}
\cvitemwithcomment{Francês}{bom}{}
\cvitemwithcomment{Russo}{fraco{}
%\section{Computer skills}
%\cvdoubleitem{category 1}{XXX, YYY, ZZZ}{category 4}{XXX, YYY, ZZZ}
%\cvdoubleitem{category 2}{XXX, YYY, ZZZ}{category 5}{XXX, YYY, ZZZ}
%\cvdoubleitem{category 3}{XXX, YYY, ZZZ}{category 6}{XXX, YYY, ZZZ}
\section{Habilidade computacional}
%\cvitem{}{}
%% Skill matrix as an alternative to rate one's skills, computer or other. 

%% Adjusts width of skill matrix columns. 
%% Usage \setcvskillcolumns[<width>][<factor>][<exp_width>]
%% <width>, <exp_width> should be lengths smaller than \textwidth, <factor> needs to be between 0 and 1.
%% Examples:
% \setcvskillcolumns[5em][][]%    adjust first column. Same as \setcvskillcolumns[5em]
% \setcvskillcolumns[][0.45][]%   adjust third (skill) column. Same as \setcvskillcolumns[][0.45]
% \setcvskillcolumns[][][\widthof{``Year''}]%     adjust fourth (years) column.
% \setcvskillcolumns[][0.45][\widthof{``Year''}]%
% \setcvskillcolumns[\widthof{``Languag''}][0.48][]
% \setcvskillcolumns[\widthof{``Languag''}]%

%% Adjusts width of legend columns. Usage \setcvskilllegendcolumns[<width>][<factor>]
%% <factor> needs to be between 0 and 1. <width> should be a length smaller than \textwidth
%% Examples:
% \setcvskilllegendcolumns[][0.45]
% \setcvskilllegendcolumns[\widthof{``Legend''}][0.45]
% \setcvskilllegendcolumns[0ex][0.46]% this is usefull for the banking style

%% Add a legend if you are using \cvskill{<1-5>} command or \cvskillentry
%% Usage \cvskilllegend[*][<post_padding>][<first_level>][<second_level>][<third_level>][<fourth_level>][<fifth_level>]{<name>}
% \cvskilllegend % insert default legend without lines
\cvskilllegend*[1em]{}% adjust post spacing
% \cvskilllegend*{Legend}%  Alternatively add a description string
%% adjust the legend entries for other languages, here German
% \cvskilllegend[0.2em][Grundkenntnisse][Grundkenntnisse und eigene Erfahrung in Projekten][Umfangreiche Erfahrung in Projekten][Vertiefte Expertenkenntnisse][Experte\,/\,Spezialist]{Legende}

%% Alternative legend style with the first three skill levels in one column
%% Usage \cvskillplainlegend[*][<post_padding>][<first_level>][<second_level>][<third_level>][<fourth_level>][<fifth_level>]{<name>}
% \setcvskilllegendcolumns[][0.6]%  works for classic, casual, banking
% \setcvskilllegendcolumns[][0.55]%  works better for oldstyle and fancy
% \cvskillplainlegend{}
% \cvskillplainlegend[0.2em][Grundkenntnisse][Grundkenntnisse und eigene Erfahrung in Projekten][Umfangreiche Erfahrung in Projekten][Vertiefte Expertenkenntnisse][Experte/Guru]{Legende}

%% Add a head of the skill matrix table with descriptions.
%% Usage \cvskillhead[<post_padding>][<Level>][<Skill>][<Years>][<Comment>]%
\cvskillhead[-0.1em]%   this inserts the standard legend in english and adjust padding
%% Adjust head of the skill matrix for other languages
% \cvskillhead[0.25em][Level][F\"ahigkeit][Jahre][Bemerkung]

%% \cvskillentry[*][<post_padding>]{<skill_cathegory>}{<0-5>}{<skill_name>}{<years_of_experience>}{<comment>}% 
%% Example usages:
\cvskillentry*{Language:}{3}{Python}{2}{linguagem que mais tive contato na minha trajetória acadêmica.}
%\cvskillentry{}{2}{Lilypond}{14}{So much sheet music! Man, I'm the best!}
\cvskillentry{}{3}{\LaTeX}{1}{Iniciante\LaTeX}
\cvskillentry*{OS:}{2}{Linux}{2}{ linux meu melhor amigo}% notice the use of the starred command and the optional 
%$\%cvskillentry*[1em]{Methods}{4}{SCRUM}{8}{SCRUM master for 5 years}
%% \cvskill{<0-5>} command
% \cvitem{\textbackslash{cvskill}:}{Skills can be visually expressed by the \textbackslash{cvskill} command, e.g. \cvskill{2}}



%\section{Extra 1}
%\cvlistitem{Item 1}
%\cvlistitem{Item 2}
%\cvlistitem{Item 3. This item is particularly long and therefore normally spans over several lines. Did you notice the indentation when the line wraps?}

%\section{Extra 2}
%\cvlistdoubleitem{Item 1}{Item 4}
%\cvlistdoubleitem{Item 2}{Item 5\cite{book2}}
%\cvlistdoubleitem{Item 3}{Item 6. Like item 3 in the single column list before, this item is particularly long to wrap over several lines.}

%\section{References}
%\begin{cvcolumns}
  %\cvcolumn{Category 1}{\begin{itemize}\item Person 1\item Person 2\item Person 3\end{itemize}}
  %\cvcolumn{Category 2}{Amongst others:\begin{itemize}\item Person 1, and\item Person 2\end{itemize}(more upon request)}
  %\cvcolumn[0.5]{All the rest \& some more}{\textit{That} person, and \textbf{those} also (all available upon request).}
%\end{cvcolumns}

% Publications from a BibTeX file without multibib
%  for numerical labels: \renewcommand{\bibliographyitemlabel}{\@biblabel{\arabic{enumiv}}}% CONSIDER MERGING WITH PREAMBLE PART
%  to redefine the heading string ("Publications"): \renewcommand{\refname}{Articles}
\nocite{*}
\bibliographystyle{plain}
\bibliography{publications}                        % 'publications' is the name of a BibTeX file

% Publications from a BibTeX file using the multibib package
%\section{Publications}
%\nocitebook{book1,book2}
%\bibliographystylebook{plain}
%\bibliographybook{publications}                   % 'publications' is the name of a BibTeX file
%\nocitemisc{misc1,misc2,misc3}
%\bibliographystylemisc{plain}
%\bibliographymisc{publications}                   % 'publications' is the name of a BibTeX file


%-----       letter       ---------------------------------------------------------
% recipient data
%\recipient{Company Recruitment team}{Company, Inc.\\123 somestreet\\some city}
%\date{January 01, 1984}
%\opening{Dear Sir or Madam,}
%\closing{Yours faithfully,}
%\enclosure[Attached]{curriculum vit\ae{}}          % use an optional argument to use a string other than "Enclosure", or redefine \enclname
%\makelettertitle



%\makeletterclosing

%\clearpage\end{CJK*}                              % if you are typesetting your resume in Chinese using CJK; the \clearpage is required for fancyhdr to work correctly with CJK, though it kills the page numbering by making \lastpage undefined
\end{document}


%% end of file `template.tex'.